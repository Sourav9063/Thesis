\chapter{Conclusion}
\label{sec:conclusion}

This thesis transcends the mere examination of Bangla IPA complexities. It delves into the language's nuances, leaving an indelible mark on both linguistic theory and practical NLP applications.

Our journey began with a thorough exploration of existing literature, dissecting the intricate web of conflicting viewpoints on Bangla IPA. This analysis meticulously constructed a robust and comprehensive IPA transcription framework specifically tailored for Bangla texts. This framework not only addresses previously identified issues but also sheds light on the previously obscured subtleties of the language.

Our contribution extends far beyond theoretical constructs. Recognizing the dearth of large-scale resources for Bangla NLP, we built a groundbreaking dataset consisting of 150,000 sentences. This monumental undertaking marks the first of its kind, offering an invaluable tool for future research and development. Its impact transcends linguistics, holding immense potential to influence and enrich the field of NLP dataset creation.

Furthermore, this dataset paves the way for advancements in Language Model (LLM) downstream tasks. By providing LLM systems with rich, nuanced Bangla language data, we unlock unparalleled opportunities for future exploration and innovation. This opens doors to an array of potential applications, propelling research forward and expanding the boundaries of language technology.

In conclusion, this thesis is not merely a culmination of efforts; it's a springboard for the future. The developed framework and pioneering dataset stand as testaments to the profound impact our work can have on linguistics and NLP. Looking ahead, we anticipate witnessing this research blossom into practical applications that enrich lives and redefine our understanding of language processing. This is not the end, but rather the beginning of a new era for Bangla language technology, and we are proud to have played a pivotal role in shaping its trajectory.


