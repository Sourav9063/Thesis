\chapter{Introduction}

\section{Introduction}

Motivated by the captivating linguistic tapestry and ubiquitous utilization of the Bengali language, with a specific focus on Bangla as the official language of Bangladesh, this research endeavors to address a discernible lacuna in the realm of Natural Language Processing (NLP) resources tailored for Bengali. Against the backdrop of a staggering population tallying 272.7 million individuals in Bangladesh and a diaspora of diverse Bengali communities dispersed globally, the exigency for a comprehensive and nuanced linguistic dataset becomes palpable. Notwithstanding the nuanced morphological variations discernible among dialects spoken in distinct regions, the salient divergences in sounds and phonology emerge as noteworthy.

\vspace{5mm}

In response to this multifaceted linguistic milieu, this thesis sets forth a pioneering model with the explicit objective of transmuting written expressions of Standard Bangla into the International Phonetic Alphabet (IPA), an established system for the phonetic and phonemic transcription of spoken languages. The impetus for this endeavor derives from a conspicuous dearth of extensive datasets in Bengali IPA, thus motivating a robust initiative to bridge this lacuna. This ambitious undertaking encompasses the creation of a groundbreaking dataset, an expansive corpus comprising no less than 70,000 words. The meticulous annotation process enlisted the expertise of three native Bangla speakers, strategically chosen for their combined acumen in linguistics and engineering.

\vspace{5mm}

The protocols governing the conversion of Bengali to IPA underwent rigorous development, orchestrated by a seasoned team possessing specialized knowledge in linguistics. This intricate framework was further subjected to scrupulous scrutiny and validation by \textbf{Dr. Syed Shahrier Rahman}, an esteemed linguist and erudite professor situated in the Department of Linguistics at the University of Dhaka. The resultant transcribed dataset, a veritable linguistic tapestry, encapsulates an extensive array of morphological forms, numbers, acronyms, abbreviations, as well as the nomenclature of places and individuals. Noteworthy in its design is the deliberate emphasis on Standard Bangla and the assimilation of borrowed words, reflecting the linguistic adaptability of native speakers. Significantly, the annotation process extends beyond mere linguistic formality, delving into the nuanced intricacies of how native speakers organically integrate Standard Bangla into their everyday discourse, particularly in the nuanced pronunciation of adapted loanwords within the contextual confines of Standard Bangla.

\vspace{5mm}

This research, animated by an earnest motivation to fill a critical void in Bengali NLP resources, culminates in the establishment of the most extensive Bangla IPA dataset to date. Beyond its sheer voluminosity, this contribution not only deepens our comprehension of Bengali phonetics but also opens up expansive vistas for groundbreaking advancements across a spectrum of NLP applications. In doing so, this thesis emerges as a beacon, illuminating pathways for further linguistic inquiry and technological innovation, all rooted in the rich tapestry of the Bengali language.


\newpage
\section{Our Contribution}

This work represents a substantial contribution to the field of \textbf{Bangla IPA Transcription}, encompassing several key advancements:

\subsection{In-Depth Study of Challenges}

We initiated a meticulous investigation of the existing issues and complexities surrounding Bangla IPA transcription. This comprehensive analysis unearthed previously unidentified challenges and provided a crucial foundation for crafting our solutions.

\subsection{Novel IPA Transcription Framework}

Driven by the insights gained from our study, we meticulously designed a novel IPA transcription framework specifically tailored for Bangla. This framework addresses the intricacies of the language, offering a robust and consistent approach to IPA representation.

\subsection{Pioneering Dataset Creation}

Recognizing the scarcity of large-scale resources, we constructed a groundbreaking dataset, aptly named \textbf{DUAL-IPA}. This sentence-level parallel corpus encompasses \textbf{160,000+} samples, serving as the first-of-its-kind resource for Bangla IPA research and NLP applications.

\subsection{Open-Source Accessibility}

Firmly believing in fostering collaboration and innovation within the research community, we have chosen to open-source the DUAL-IPA dataset under the \textbf{CC BY-SA 4.0 license}, making it freely accessible to researchers and practitioners worldwide.

\subsection{National Competition}

The impact of our work extends beyond academic realms. We proudly collaborated with the Institute of Information Technology (IIT), University of Dhaka, to organize a national-level competition leveraging the DUAL-IPA dataset. Witnessing the participation of 104 teams, with over 50 actively utilizing our model, serves as a testament to the real-world application potential and community engagement fostered by this research.


\hspace{0.5cm}Through these combined efforts, we believe we have made a significant contribution to advancing the field of Bangla IPA transcription. By addressing existing challenges, providing innovative solutions, and fostering open-source accessibility, we pave the way for future advancements in language processing and related NLP applications for Bangla.






