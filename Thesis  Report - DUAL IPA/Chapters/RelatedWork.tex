\chapter{Related Work}  
%   5 page
\section{International Phonetic Alphabet (IPA)}

The requirement of such a model to transcribe the Bengali language to IPA requires a phonetic transcription scheme to represent the transcription and the pronunciation patterns for the language. The International Phonetic Alphabet (IPA) stands as the sole standard for phonetic writing systems, accounting for its significance in the scientific examination of a language's phonetics. Regardless of the language in question, the International Phonetic script predominantly relies on Roman characters as well as incorporates modified elements from diverse scripts like Greek to convey phonetic notation. The IPA-provided symbols such as (\ipa{t}, \ipa{ɛ}, \ipa{ʃ}, \ipa{k}, \ipa{ ̪ }) are to be used for even those language that does not employ the Roman alphabet, such as Bangla, Hindi, Japanese, or Korean. 

\vspace{5mm}

Since its establishment in 1886, the International Phonetic Association has been concerned with developing a system of symbols that maintains a balance between usability and inclusivity, which includes the wide variety of sounds present in languages all over the world. ~\cite{international1999handbook}The main purpose of IPA is to represent specific speech sounds rather than the abstract linguistic units known as phonemes, although it is also used for phonemic transcription. The IPA follows a common policy of using one letter for each segment. As a result, two letters are not put together to represent one single sound. For example, ‘shine’ - in this word ‘sh’ is used to convey one single sound. The IPA doesn't usually provide separate characters for sounds that aren't differentiated in known languages. Both broad and narrow transcriptions can be used using the IPA.

\newpage
\subsection{Bangla Vowels}
~\citet{chatterji1921bengali} used ~\citet{jones1922outline}'s cardinal vowel system to explain the Bangla vowel system. He claimed that the Bangla language has seven primary vowels \textbengali{ই}\ipa{/i/}, \textbengali{এ}\ipa{/e/}, \textbengali{অ্যা}\ipa{/æ/}, \textbengali{আ}/\ipa{a/}, \textbengali{ও}\ipa{/o/}, \textbengali{অ}\ipa{/ɔ/}, and \textbengali{উ}\ipa{/u} along with their corresponding nasal counterparts \ipa{/ĩ ẽ æ̃ ã õ ɔ̃ ũ/}. Chatterji also noted that Bangla vowels are generally articulated in a lax manner, imparting the characteristic 'timbre' to the vowel system.  ~\citet{morshed1997bhashatatwa} categorized the vowels as \ipa{/i, u, e, o, ae, ɔ, and a/}, including two high, two high-mid, two low-mid, and one low vowel. ~\citet{ali2001dhanibijnaner} investigated vowel contrasts, defining phonological properties, and reported the same number of vowels, with a subtle distinction. He employed the symbol \ipa{/ɛ/} to represent the vowel \ipa{/æ/} as described by~\citet{morshed1997bhashatatwa}.

%\begin{table}[!htbp]
%    \centering
%    \resizebox{0.40\textwidth}{!}{%
%    \begin{tabular}
    % {p{20mm}p{15mm}p{15mm}p{15mm}}
%    {lccc}
    
%        \hline
%         & {\scriptsize Front} & {\scriptsize Central} & {\scriptsize Back} \\
%       \hline
        
%        {\scriptsize High} & \ipa{i} & & \ipa{u} \\
%        \hline

%        {\scriptsize High-mid} & \ipa{e} & & \ipa{o} \\
%        \hline

%        {\scriptsize Low-mid} & \ipa{æ} & & \ipa{ɔ} \\
%        \hline

%        {\scriptsize Low} & & \ipa{a} & \\
%        \hline

%    \end{tabular}%
%    }
%    \caption{Bangla vowels by~\citet{morshed1997bhashatatwa}}
%    \label{tab:bangla_vowels}
%\end{table}

In a separate study, ~\citet{hai1964dhwonibijnan} analyzed the vowels of Standard Bangla using the concept of cardinal vowels. He claimed that there are eight vowels \textbengali{ই}\ipa{/i/}, \textbengali{এ}\ipa{/e/}, \textbengali{অ্যা}\ipa{/æ/},  \textbengali{আ}\ipa{/a/a}, \textbengali{ও}\ipa{/o/} \textbengali{ও'}\ipa{/o’/}, \textbengali{অ}\ipa{/ɔ/}, and \textbengali{উ}\ipa{/u/} in the Bangla language. He categorizes \textbengali{ই}\ipa{/i/}, \textbengali{এ}\ipa{/e/}, \textbengali{অ্যা}\ipa{/æ/} as front vowel and \textbengali{ও}\ipa{/o/}, \textbengali{ও'}\ipa{/o’/}, \textbengali{অ}\ipa{/ɔ/}, and \textbengali{উ}\ipa{/u/} as back vowel. In contrast to ~\citet{morshed1997bhashatatwa}, Hai did not classify the Bangla vowel \textbengali{আ}\ipa{/a/a} as occupying a central position. He explained that the Bangla \textbengali{আ}/a/ sound differs from the neutral quality of the English \ipa{/a/} and is distinct from the Urdu close \ipa{/ə/} sound. Instead, he characterized it as an open vowel. Hai also pointed out the presence of an additional vowel in the Bangla vowel system, denoted as \ipa{/o'/}. He explained that when producing the \ipa{/o'/} sound, the lips are slightly less rounded compared to the \ipa{/o/} sound. However, there isn't a significant difference in the gap between the jaws, and the back of the tongue is not raised as much as it is when articulating the \ipa{/o/} sound. This led him to term it as yotized o \ipa{(oʸ)}, known in Bangla as \textbengali{অভিশ্রুত} \ipa{/obʱɪsɾut̪o/} \textbengali{ও} \ipa{/o/} or \textbengali{ও'} \ipa{/o'/}. This observation was supported by ~\citet{huq2002bhasha}. An example provided for this distinction is between \textbengali{বিয়ের ক'নে}\ipa{/bɪʲeɾ ko’ne/} and \textbengali{ঘরের কোণে} \ipa{/gʱɔɾeɾ kone/}. Nevertheless, it's worth noting that there is limited empirical evidence to support this concept. On the contrary, the claim that the number of vowels is seven is backed by Pobitro Sorkar (1992) and Puny Sloka Ray (1997) as noted in ~\citet{ali2001dhanibijnaner}.


\subsection{Bangla Semi-Vowels}
%The debate regarding the number of semivowels in Bangla is a subject of discussion. 
According to ~\citet{chatterji1921bengali} and ~\citet{sen1993itibritta}, there are two Bangla semivowels, namely \textbengali{অন্তস্থ ব}\ipa{/w/} and \textbengali{অন্তস্থ য়} \ipa{/y/}. ~\citet{hai1964dhwonibijnan} contends that there are three semivowels: \textbengali{অন্তস্থ ব}\ipa{/w/}, \textbengali{অন্তস্থ য়} \ipa{/y/}, and \textbengali{অন্তস্থ ই}\ipa{/i/}. ~\citet{morshed1997bhashatatwa} argues that while \textbengali{অন্তস্থ ব}\ipa{/w/} and \textbengali{অন্তস্থ য়} \ipa{/y/} are considered semivowels in English, they do not possess similar status in Bangla. A different perspective was presented by ~\citet{ferguson1960phonemes}, who claim that there are four semivowels: \ipa{/i e o u/}. It is noted in ~\citet{ali2001dhanibijnaner} that this assertion was supported by Pobitro Sharker and Ghonesh Boshu (1998). Along with the \textbengali{ই}\ipa{/i̯/}, \textbengali{উ}\ipa{/u̯/}, and \textbengali{ও}\ipa{/o̯/}, there is a fourth semi-vowel which is \textbengali{এ}\ipa{/e̯/} that is found at the end of the word in the form of \textbengali{‘য়’} such as \textbengali{হয়}\ipa{/hɔ͡e̯/}, \textbengali{যায়}\ipa{/ja͡e̯/}~\cite{ali2001dhanibijnaner}.




\subsection{Bangla Diphthongs}
~\citet{sen1993itibritta} noted that the Bangla has two diphthongs: \textbengali{ঐ}(\ipa{oɪ}) and \textbengali{ঔ}(\ipa{ou}). These combinations of two sounds do not fit the conventional definition of diphthongs but are represented in written form. In linguistic terms, they are referred to as digraphs~\cite{ali2001dhanibijnaner}. On the contrary, ~\citet{chatterji1921bengali} claimed that there are 25 diphthongs in standard Bangla. ~\citet{hai1964dhwonibijnan} asserted that there are a total of 31 diphthongs, categorizing them into 19 regular and 12 irregular ones. However, he also once argued that there are only 18 diphthongs, as noted by ~\citet{ali2001dhanibijnaner}, who in turn asserts that there are 17 diphthongs in Bangla. The government-approved IPA website acknowledges the regular 19 diphthongs, but they have used the diphthong \ipa{/ui̯/} two times and did not consider the \ipa{/eo̯/} diphthong.


\subsection{Bangla Consonants}
There have been numerous past studies, primarily rooted in articulatory phonetics, that have examined the articulatory and acoustic characteristics of Bangla consonants. It is described in~\citet{hai1964dhwonibijnan} that Bangla consonant has 20 stops, 7 fricatives, 4 nasals, 1 lateral, 1 trill, 2 flaps, and 1 glide; totaling 36 consonants. ~\citet{hai1964dhwonibijnan} claims that there's only on phone close to  \ipa{/ʃ/} in Bangla. ~\citet{huq2002bhasha} presented a slightly different categorization of a total of 35 consonants, presenting 21 stops, 5 fricatives, 3 nasals, 1 lateral, 1 trill, 2 flaps, and 2 glides. ~\citet{morshed1997bhashatatwa} stated that Bangla includes 20 stops, 4 nasals, 4 fricatives, 1 lateral, and 2 flaps, totaling 31 consonants. On the other hand, ~\citet{ali2001dhanibijnaner} argued that Bangla has 20 stops, 3 nasals, 3 fricatives, 1 lateral, 2 flaps, 1 trill, and 2 glides, resulting in a total of 32 consonants.



