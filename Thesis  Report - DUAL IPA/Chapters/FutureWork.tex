\chapter{Future Plans and Perspectives}

The creation of this dataset and initial model for Bengali IPA transcription marks a significant step forward in the field of Bengali NLP. However, it also opens up exciting avenues for future exploration and further development. Here, we outline some potential directions for expanding upon this work:


\section{Expanding the Dataset}

\hspace{0.5cm}\textbf{Size:} Increasing the size and diversity of the dataset will enhance the model's generalizability and robustness. This could involve collecting data from different dialects, age groups, and speaking styles.

\textbf{Speech Recordings:} Incorporating speech recordings alongside IPA annotations would enable research on tasks like automatic speech recognition and text-to-speech synthesis for Bengali.

\textbf{Multilingualism:} Including translations of the Bengali text could facilitate cross-lingual research and applications.

\section{Model Development}

\hspace{0.5cm}\textbf{Advanced Architectures:} Experimenting with more sophisticated model architectures like Transformer-based approaches could further improve performance on various NLP tasks.

\textbf{Task-Specific Models:} Developing specialized models for different tasks like machine translation, sentiment analysis, or summarization could offer practical applications.

\textbf{Multimodal Learning:} Integrating audio and text data within a single model could lead to advancements in speech-related NLP tasks.

\section{Applications and Impact}

\hspace{0.5cm}\textbf{Language Learning:} The dataset and model could be used to develop language learning tools and resources for Bengali.

\textbf{Accessibility:} The technology could be adapted to create speech-to-text applications for individuals with disabilities.

\textbf{Cultural Preservation:} The project can contribute to the preservation and documentation of Bengali dialects and regional variations.

\section{Collaboration and Community Building}

\hspace{0.5cm}\textbf{Sharing the Dataset:} Making the dataset publicly available would encourage further research and collaboration within the NLP community.

\textbf{Open-Sourcing the Model:} Open-sourcing the model would allow others to build upon it and create new applications.

\textbf{Community Engagement:} Building a community around Bengali NLP could attract researchers, developers, and enthusiasts to contribute to the field's growth.


\section{LREC Coling - 2024}

In addition to the aforementioned future directions for research, I am thrilled to announce that this work has been submitted for presentation at the prestigious \textbf{LREC-COLING 2024 – The 2024 Joint International Conference on Computational Linguistics, Language Resources and Evaluation Lingotto Conference}, scheduled for 20-25 May 2024 in Torino (Italia). This esteemed international forum provides an invaluable platform to engage with leading researchers and practitioners in computational linguistics, language resources, and evaluation. Presenting our findings at LREC-COLING 2024 offers an exciting opportunity to gather expert feedback, foster further collaboration, and contribute to the advancement of the field. We anticipate fruitful discussions and a chance to learn from the diverse perspectives of the conference attendees, ultimately enriching our research and propelling it towards even greater impact.