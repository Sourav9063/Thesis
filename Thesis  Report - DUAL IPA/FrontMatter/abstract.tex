% \begin{center}
%         \huge
%         \textbf{{Abstract}}\\
%         \vspace{1cm}
%     \end{center}
 
 
%  \begin{centering}
%  \justify
% This will be the abstrack

% \end{centering}

\begin{abstract}
% Bengali is one of the most spoken languages in the world with over 300 million native and 37 million international speakers; and with printed Bengali documents that date back to the 1800s.
% Despite being a major language, research into Bengali optical character recognition has been slow. 
% Although recent advancements in deep learning has helped progress in the field of recognition, analysis of document layout remains an unsolved problem. 
% \textcolor{red}{rewrite this sentence: In this paper, we released a Bengali complex document layout dataset annotated with 4 segmentation labels -- text box, paragraph, image and table for all of the 40,000 document samples.} 
% We also provide word and sentence level segmentation with coordinates through machine-labelling. 
% Sampled from diverse sources starting from early 1800s, this dataset contains old newspapers, magazines, literature books, government documents, land documents, textbooks etc. We also present a benchmark model on the data in this paper.

% \keywords{First keyword  \and Second keyword \and Another keyword.}


This thesis presents a significant contribution to the field of Natural Language Processing (NLP) by introducing a novel and comprehensive Bengali dataset comprising 150,000 sentences transcribed in the International Phonetic Alphabet (IPA). Bengali, characterized by its intricate phonological features, stands as a promising subject for advanced NLP research. The meticulously curated dataset aims to capture the subtle phonetic nuances of Bengali, serving as a valuable resource for training and evaluating NLP models. The abstract outlines a detailed statistical analysis, annotation methodologies, and explores potential applications within the NLP domain, emphasizing the dataset's pivotal role in advancing the understanding of the Bengali language.

The discussion delves into the dataset's significance, pushing the boundaries of Bengali language comprehension and broader NLP research. Furthermore, the abstract highlights the dataset's versatility by showcasing its potential in enhancing various NLP tasks such as speech recognition, language modeling, sentiment analysis, and named entity recognition. This research not only contributes to the existing body of knowledge but also provides a foundation for future developments in Bengali NLP research, underscoring the importance of the dataset in fostering advancements in language technology.

\vspace{5mm}
{\textbf{Keywords:} IPA, Bengali, Linguistics}

\end{abstract}

